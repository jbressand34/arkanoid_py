\section{Gestion du projet}
  Nous avons décidé d'utiliser Python3 pour notre projet. Cela nous a occasioné des problèmes de compatibilité, de nombreux modules étant prévus pour Python2. Nous verrons ici la liste des outils abordés, notre organisation et les problèmes d'ordre humain rencontrés dans le déroulement du projet.
  
\subsection{Listes des outils abordés}
  \begin{enumerate}
  %\renewcommand{\labelitemi}{$\bullet$}
     \item[\bf Tkinter] Bibliothèque graphique du langage python, permettant la création d'interfaces graphiques et de différentes sortes de widgets.
    \item[\bf Pygame] Bibliothèque graphique de python plus riche que tkinter. N'apportant rien de crucial et s'étant déjà familiarisé avec tkinter, nous ne l'avons pas utilisé.
    \item[\bf Pymunk] Trop complexe : Outil de gestion de contraintes et collisions pour python. Sa complexité et nos délais impartis nous ont empeché de pouvoir l'utiliser.
    \item[\bf PIL] Bibliothèque de chargement d'image. Obsolète.
    \item[\bf Pillow] Fork de PIL non obsolète. Nous n'avons pas réussi à l'utiliser.
    \item[\bf Pickle] Bibliothèque python permettant de transformer un objet python en code binaire et vice versa. Elle est utilisée dans la gestion des scores et des niveaux.
    
    \item[\bf Git] Outil de gestion de version de fichiers. Nous avons notamment eu recours aux :
	\begin{description}
      \item[Branchs] Nous avons utilisés une branche de développement, et une branche contenant les build stables de notre projet. Une capture d'écran de notre working tree est jointe en annexe.
      \item[Wiki] Nous avons utilisé le wiki interne au git pour partager l'ensemble des informations relatives au projet : dates des réunions, comptes rendus, choses à faire dans la semaine, idées, ...
      \item[Issues] Nous les avons utilisés pour rendre compte des bugs à corriger, des features et éventuelles améliorations à ajouter. Une capture d'écran de nos issues est jointe en annexe.
    \end{description}
    \item[\bf Jabber, Skype]  Jabber est un protocole de communication. Suite à des difficultés relatives à son utilisation pour faire des conférences, nous avons utilisé skype.
    \item[\bf Geogebra] Outil de modélisation géométrique. Il nous a permis de modéliser aisément les fonctions associés au vecteur vitesse de la balle après rebond sur la barre.
    \item[\bf Framapad] Service d'édition de texte collaborative en ligne.
    \item[\bf Ganttproject] Logiciel d'édition de diagramme de Gantt.
    \item[\bf LaTeX]
  \end{enumerate}
\subsection{Organisation du travail} % Gantt
	Le travail a été divisé en trois axes majeurs : analyse, conception de notre programme et implémentation.
	
	La première phase s'est notamment composée de réunions pour aligner notre vision du projet. Nous avons d'abord fait de nombreuses suggestions, nous avons débattu à leurs sujets et à terme nous avons posé le cahier des charges avec notre encadrant. Au fur et à mesure de l'apprentissage des outils liés au projet, nous avons allégé notre cahier des charges dans un soucis de réalisme.
	La phase de conception a également necessité plusieurs réunions pour se mettre d'accord sur la structure du projet. L'élaboration de nos algorithmes s'est fait parallèlement à leur implémentation.
	La troisième phase s'est composée du developpement de la version objet suivie de l'impérative.
	
	Un diagramme de Gantt et la liste des {\em work packages} associés sont joints en annexe.
	
    
\subsection{Gestion des ressources humaines}
	Ce projet ayant été une première expérience de travail collaboratif, des tensions sont apparues. Elles ont été liées à des niveaux d'implication différents ou à des divergences de point de vues. Aussi les tâches ont été distribuées au sein du groupe sur la base du volontariat.
	
	Les niveaux d'implication différents des membres du groupe ont été la source d'un gros retard de la version impérative sur la version objet. 

	On a observé un autre désaccord relatif au mode campagne, deux membres du groupe ne s'étant pas compris sur sa définition. D'autres désaccords de moindre ampleur ont également ponctué notre travail.