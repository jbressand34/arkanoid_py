\section{Problématique}

Le sujet de notre Projet informatique de L2 est la production d'un jeu de type casse brique, à  l'image d'\href{http://fr.wikipedia.org/wiki/Arkanoid}{Arkanoid}.
Son but pédagogique est de fournir un programme à étudier à des étudiants en parcours Physique.
La plupart des logiciels utilisés par les physiciens sont codés en Python. Une demande du client est d'utiliser ce langage pour notre projet.
Aussi il nous a été demandé d'utiliser la bibliothèque tkinter, permettant une gestion simple d'éléments graphiques. Dans un souci de simplicité, on a choisi de manipuler des rectangles. 
Le type de jeu de notre projet nous permet d'aborder de façon pédagogique un ensemble de concept de programmation, tels que
les structures de contrôles, les strucures de données, la manipulation des types de base python (liste, tuple, dictionnaire) et des fichiers ainsi que la gestion des exceptions.
Conformément à la demande du client, nous rendrons notre projet sous deux formes différentes, impérative et orientée objet.
                                   
Arkanoid est un jeu d'arcade conçu par Akira Fujita, développé et édité par Taito. Ce jeu sorti en 1986 sur borne d'arcade est un casse brique. C'est un jeu de dextérité où l'on contrôle un vaisseau spatial en forme de barre. Ce jeu possède une histoire simple : après la destruction par une attaque extraterrestre du vaisseau Arkanoid, un vaisseau rescapé nommé Vaus est envoyé dans une autre dimension pour se venger et détruire Doh l'alien responsable de l'attaque. 
Le jeu se compose d'une balle, d'une barre et de briques. Le but est de casser les briques grâce à la balle que l'on fait rebondir. Le joueur perd lorsqu'il perd la balle.
